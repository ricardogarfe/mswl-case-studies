\documentclass[11pt]{article}
\usepackage[utf8]{inputenc}
\usepackage[spanish]{babel}
\usepackage[parfill]{parskip}
\usepackage{listings}
\lstdefinestyle{Shell}{delim=[il][\bfseries]{BB}}

\title{\textbf{MySQL}}
\author{Ricardo Garc\'ia Fern\'andez}
\date{\today}
\begin{document}

\maketitle

\section{Introducci\'on}

\emph{MySQL}: Es un sistema de bases de datos relacional, que se distingue por:

\begin{itemize}
	\item Base de datos relacional.
	\item Multiplataforma.
	\item Software libre desde el año 2000 con doble licenciamiento, con licencia GPL y otras licencias comerciales.
	\item Compatibilidad SQL
	\item Triggers, functions, stored procedures, GIS, replication, clustering, \ldots
    \item Soporta varios motores de almacenamiento (MyISAM, InnoDB, MERGE, MEMORY ...)
\end{itemize}

\section{MySQL}\label{sec:mysql}

\par Para ponernos en contexto, se va a presentar un poco de historia alrededor de MySQL.

\par Fue desarrollado por Michael Widenius y David Axmark, quienes en 1995 fundaron la compa\~n\'ia \emph{MySQL AB}. A partir de 1995 la compañ\'ia fue creciendo desde su primera vesi\'on hasta que en el a\~no 2000 se public\'o con licencia GPL.

\par En el a\~no 2005 se publica MySQL 5.0. A partir de esta fecha empiezan los verdaderos cambios desde fuera de MySQL:

\begin{itemize}
	\item En 2008 Sun compr\'o MySQL.
	\item En 2008 Oracle compr\'o Sun. Varios t\'ecnicos crearon un ``fork``. MariaDB\footnote{https://mariadb.org/} es uno de los forks m\'as famosos de MySQL.    
	\item En 2011 Oracle empieza a vender extensiones comerciales de MySQL.
\end{itemize}

\par Podemos ver que actualmente se encuentra en la versión \textbf{5.5.28}.

\section{Instalaci\'on}

\par Se puede instalar f\'acilmente desde el mismo gestor de paquetes en una distribuci\'on basada en Debian, como es el caso de Ubuntu:

\begin{lstlisting}[style=Shell]
ricardo@ricardo-ubuntu:~$ sudo apt-get install mysql-client
ricardo@ricardo-ubuntu:~$ sudo apt-get install mysql-server
\end{lstlisting}

Se han de seguir paso a paso las instrucciones a la hora de configurar MySQL.

\begin{itemize}
	\item Contrase\~na del usuario root.
\end{itemize}

\section{Command Line client}
\label{sec:cl-client}

\par Ahora ya podemos acceder al cliente MySQL a través del cliente de comandos.

\par Command:
\begin{lstlisting}[style=Shell]
mysql [<options>] database
\end{lstlisting}

\par Common options:
\begin{lstlisting}[style=Shell]
--user, -u: MySQL user name
--password, -p: user's password
--host, -h: connect to the server on the given host
\end{lstlisting}

\par Entrar en el cliente de la base de datos:
\begin{lstlisting}[style=Shell]
ricardo@ricardo-ubuntu:~$ mysql -u root -padmin
Welcome to the MySQL monitor.  Commands end with ; or \g.
Your MySQL connection id is 193
Server version: 5.5.28-0ubuntu0.12.04.3 (Ubuntu)

Copyright (c) 2000, 2012, Oracle and/or its affiliates. 
All rights reserved.

Oracle is a registered trademark of Oracle Corporation 
and/or its affiliates. Other names may be trademarks 
of their respective owners.

Type 'help;' or '\h' for help. Type '\c' to clear
 the current input statement.
mysql>
\end{lstlisting}

\par Podemos empezar a familiarizarnos con los comandos básicos:

\begin{enumerate}
    \item Crear una base de datos::
\begin{lstlisting}[style=Shell]
mysql> create database [nombre\_de\_bbdd];
\end{lstlisting}

    \item Acceder a ella::
\begin{lstlisting}[style=Shell]
mysql> use [nombre\_de\_bbdd];
\end{lstlisting}
    
    \item Eliminar la tabla::
\begin{lstlisting}[style=Shell]
mysql> drop [nombre\_de\_bbdd];
\end{lstlisting}
    
    \item Mostrar tablas::
\begin{lstlisting}[style=Shell]
mysql> show tables;
\end{lstlisting}

    \item Descripción de una tabla::
\begin{lstlisting}[style=Shell]
mysql> desc [nombre\_de\_tabla];
\end{lstlisting}

\end{enumerate}

\section{Backups}
\label{sec:backups}

\par Se pueden aplicar diferentes tipo de filtros a la hora de crear un backup de la base de datos:

\begin{itemize}
	\item BBDD completa.
	\item Determinadas Tablas.
	\item etc\ldots
\end{itemize}

\section{GUI Tools}
\label{sec:gui-tools}

\par MySQL compr\'o \emph{DBDesigner4}\footnote{http://fabforce.net/dbdesigner4/} para desarrollar \textbf{MySQLWorkbench}\footnote{http://www.mysql.com/products/workbench/}.

Tambi\'en podemos optar por la elecci\'on de \emph{PhpMyAdmin}\footnote{http://www.phpmyadmin.net/home\_page/index.php} a trav\'es de un interfaz web.

\section{Peculiaridades}
\label{sec:peculiaridades}

\par En MySQL, los tipos de Datos Gis\footnote{http://dev.mysql.com/doc/refman/5.0/es/spatial-extensions.html} no estaban soportados con anterioridad. Este tipo de datos ofrece valores que representan a un elemento geogr\'afico, donde por elemento geogr\'afico entendemos; \emph{cualquier cosa en el mundo que tenga una ubicaci\'on}.

\begin{thebibliography}{9}

    \bibitem{guide-mysql}
    The Definitive Guide to MySQL 5, Apress,\\
    Kofler, Michael,\\
    2005.

    \bibitem{pro-mysql}
    Pro MySQL, Apressddison-Wesley,\\
    Kruckenber and Pipes.,\\ 
    2005.

\end{thebibliography}

\end{document}
