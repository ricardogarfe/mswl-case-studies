\documentclass[11pt]{scrartcl}
\usepackage[utf8]{inputenc}
\usepackage[spanish]{babel}

\title{\textbf{R}}
\subtitle{Language and environment for stadistical computing graphics}
\author{Felipe Ortega}
\date{\today}

\begin{document}

\maketitle

\section{Introducci\'on}

\textbf{R} es un lenguaje y un entorno para datos y gr\'aficos estad\'isticos.
Se basa en el Lenguaje \textbf{S}. 
Ross Ihaka y Robert Gentleman.
Licencia  GNU v2.

\section{Propiedades}

Se est\'a constituyendo como est\'andar 'de facto'.
Bibliotecas gestionadas a trav\'es de un repositorio para facilitar la integraci\'on.

\section{Instalaci\'on}

	bash> apt-get install r-base r-cran-rmysql

\subsection{GUI}

R-commander install:

	bash> R
	R> install.packages('Rcmdr', dep = T)
	R> library(Rcmdr)

\section{Commandos}

Ayuda:

	R> help(functionname)
	R> help.search(''word'')
	R> apropos(''word'')

Inside my library:

	R> library(help=MASS)

\begin{thebibliography}{9}

\bibitem{rinformation}
  The Comprehensive R Archive Network,\\
  http://www.cran.r-project.org/

\end{thebibliography}

\end{document}