\documentclass{scrartcl}

\title{\textbf{Intro to Shells}}
\subtitle{}
\author{Ricardo Garc\'ia Fern\'andez}
\date{\today}

\begin{document}

\maketitle

\section{Introducci\'on}

El inicio de las shells, el arranque del PC.

\section{Que es una shell}

Una \textbf{shell} es una aplicaci\'on m\'as que se ejecuta cuando un usuario inicia una sesi\'on.

\subsection{Hist\'orico de Shells}

Una peque\~na historia a trav\'es de la evoluci\'on de las shells.

\subsubsection{Thompson Shell}

No variables, scripting, autocompletado pero es \emph{la precursora} las futuras shells de linux.
Escrita por Kent Thompson creador de Unix mediante el lenguage \textbf{C}\footnote{http://v6shell.org/}.

Licencia BSD.

\subsubsection{Bourne Shell}

Desarrollada por Stephen Bourne en 1977 en AT\&T programada en \textbf{C}.
Fue criticada a pesar de que se convirti\'o en la estandar para los sistemas Linux.

Licencia BSD.

\subsubsection{C shell}

Bill Joy, tambi\'en desarrollada en \textbf{C}.

historial
completado ficheros
aliases
La sitanxis era muyparecida a C y en este aspecto 

Pero no se pod\'ian utilizar funciones.

Licencia BSD.

\subsubsection{Tenex C shell: tcsh}

TENEX fue un sistema operativo para las computadoras PDP-10(referencia).

Autocompletado de comandos.
Uso avanzado del historial.

Licencia BSD.

\subsubsection{Korn shell}

Posix 1003.2 (referencia).
Compatibilida con Bourne Shell.
Software propietario hasta el a\~no 2000.

Alternativas libres:

pdksh: shell por defecto de OpenBSD.
mksh: shell por defecto de Android.

Licencia BSD.

\subsubsection{Bourn-again sehll}

Bash: actualmente la shell principal de los sistemas Linux.

Incluye ideas de sh, ksh, csk.
Programar funciones.
Argunmentos por defecto.
Scripts de inicio.

Licencia GPLv3.

\subsubsection{Z shell: zsh}

Paul Flastad en la Universidad de Princeton en 1990.\\
El nombre procede de Zhong Shao (zsh) el tutor de su tesis.\\
Es una shell 'cool'.\\
Licencia MIT.

\section{Instalaci\'on de shells}

\begin{itemize}
    \item csh
    \item tcsh
    \item zsh - probar.
    \item pdsh
\end{itemize}

\section{Scripting}

Las shells son diferentes, es decir la sintaxis puede variar por lo que hay que pensar en desarrollar para m\'ultiples plataformas.

\end{document}