\documentclass{scrartcl}
\usepackage[utf8]{inputenc}
\usepackage[parfill]{parskip}
\usepackage{graphicx}
\usepackage{float}
\usepackage{fancyvrb}
\usepackage{listings}

\lstdefinestyle{Shell}{delim=[il][\bfseries]{BB}}

\title{\textbf{Intro to Shells}}
\author{Ricardo Garc\'ia Fern\'andez}
\date{\today}

\begin{document}

\maketitle

\section{Introducci\'on}

El inicio de las shells, el arranque del PC.

\begin{enumerate}
	\item Arrancamos el PC
	\item La BIOS toma el control: comprueba dispositivos y busca instrucciones en el MBR
	\item El MBR apunta al cargador de arranque (GRUB o LILO)
	\item El cargador de arranque inicia el kernel
	\item Lo primero que hace el kernel es ejecutar el proceso init
	\item En funci\'on del nivel de ejecuci\'on, se van lanzando m\'as scripts y servicios
	\item Por ultimo, aparece la pantalla de login
\end{enumerate}

Aqu\'i empieza nuestro primer contacto con una shell que en castellano significa \emph{caparaz\'on}.

\section{Que es una shell}

Una \textbf{shell} es una aplicaci\'on m\'as que se ejecuta cuando un usuario inicia una sesi\'on que permite la interacci\'on entre el usuario y el sistema operativo.

Una peque\~na shell de ejemplo implementada con python:

\begin{lstlisting}[style=Shell]
python pyshell.py
#!/usr/bin/env python
import os
prompt = "$ "
while True:
line = raw input(prompt)
os.system(line)
\end{lstlisting}

\section{Hist\'orico de Shells}

Una peque\~na historia a trav\'es de la evoluci\'on de las shells a trav\'es del tiempo.

\subsection{Thompson Shell}

No variables, scripting, autocompletado pero es \emph{la precursora} las futuras shells de linux.
Escrita por Kent Thompson creador de Unix mediante el lenguage \textbf{C}\footnote{http://v6shell.org/}.

Licencia BSD.

\subsection{Bourne Shell}

Desarrollada por Stephen Bourne en 1977 en AT\&T programada en \textbf{C}.
Fue criticada a pesar de que se convirti\'o en la estandar para los sistemas Linux.

Licencia BSD.

\subsection{C shell}

Bill Joy, tambi\'en desarrollada en \textbf{C}.

A\~nade como mejoras:
\begin{itemize}
	\item historial
	\item completado ficheros
    \item aliases
\end{itemize}

\par La sitanxis es muy parecida a C y en este aspecto, ayud\'o a su f\'acil adopci\'on. 

\par No se pod\'ian utilizar funciones, este era su punto negativo.

\par Licencia BSD.

\subsection{Tenex C shell: tcsh}

\emph{TENEX} fue un sistema operativo para las computadoras PDP-10\footnote{http://dl.acm.org/citation.cfm?id=361271}.

\par Las funcionalidades a destacar por encima de las dem\'as:
\begin{itemize}
	\item Autocompletado de comandos.
	\item Uso avanzado del historial.
\end{itemize}

\par Licencia BSD.

\subsection{Korn shell}

\par Posix 1003.2\footnote{http://docstore.mik.ua/orelly/unix3/korn/appa\_01.htm}.

\par Esta shell destaca por la compatibilidad con Bourne Shell. Fue software propietario hasta el a\~no 2000 en donde se public\'o bajo la licencia BSD.

\par Alternativas libres:
\begin{itemize}
	\item pdksh: shell por defecto de OpenBSD.
    \item mksh: shell por defecto de Android.
\end{itemize}

\subsubsection{Bourn-again sehll}

\par Bash: actualmente la shell principal de los sistemas Linux, por lo tanto m\'as conocida por los usuarios.

\par Est\'a compuesta por ideas de sh, ksh, csk.

\par Destaca por la facilidad para programar funciones. Se pueden definir argumentos por defecto y a\~nadir Scripts de inicio.

\par Est\'a publicada bajo la licencia GPLv3.

\subsubsection{Z shell: zsh}

\par Creada por Paul Flastad\footnote{http://www.falstad.com/} en la Universidad de Princeton en 1990. El nombre \emph{zsh} procede de Zhong Shao\footnote{http://cs-www.cs.yale.edu/homes/shao-zhong/}, el tutor de su tesis.

\par Es una shell 'cool' bajo la licencia MIT.

\section{Instalaci\'on de shells}

\begin{itemize}
    \item csh
    \item tcsh
    \item zsh
    \item pdsh
\end{itemize}

\section{Scripting}

\par Las shells son diferentes, es decir, la sintaxis puede variar por lo que hay que pensar en desarrollar para m\'ultiples plataformas.

\section{Multi shell}

\par Como hemos podido observar, existen distintos tipos de Shell. Por lo que se puede dedicar un trabajo espec\'ifico a cada una de ellas. Trabajando entre las mismas intercambiando informaci\'on y as\'i, obteniendo un mayor rendimiento para cada objetivo.

\par Aunque con un poco m\'as de conocimiento, la interoperabilidad en las Shells, no tiende a extenderse. La gente utiliza la que m\'as se adecua a su trabajo y var\'ia en funci\'on de las necesidades.

\end{document}