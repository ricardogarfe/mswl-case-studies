\documentclass[11pt]{article}
\usepackage[utf8]{inputenc}
\usepackage[spanish]{babel}

\title{\textbf{Translations in Libre Software}}
\author{Ricardo Garc\'ia Fern\'andez}
\date{}
\begin{document}

\maketitle

\section{Introducci\'on}

Proceso de traducción de un proyecto de software paso a paso:
\begin{itemize}
    \item Objetos a internacionalizar.
    \item Internacionalización.
    \item Localización.
    \item Revisión, Mantenimiento y Quality Assurance.
\end{itemize}

\section{Objetos a internacionalizar}

Saber que es lo que hay que internacionalizar y como.

\section{Internacionalización}



\section{Localización}

Herramientas de localización.

\begin{itemize}
    \item Investigar \emph{GNU Gettext}, existente desde los años 80.
    \item ¿ Android ? Spring... etc, etc, etc...
    \item Gnome.
\end{itemize}

\emph{Standalone translation tools}; Poedit, Gtranslator, Lokalize, Virtaal.
    Estas herramientas proporcionan unos test de calidad para las traducciones hechas por el usuario por lo que comprueba que la traducción corresponde (como) a la realidad. \textbf{¿Autocorrección de palabras?}

\emph{Translation Web platforms}; Pootle, Launchpad, Transifex, Weblate.
Se coordinan con el repositorio de código online para gestionar los ficheros.

\emph{Módulos}; Wordpress, Drupal, ¿Blogger?

\section{Revisión, mantenimiento y QA}

Coordinación entre los traductores y los desarrolladores para obtener un resultado completo a la hora de lanzar el producto.
Establecer las guías básicas para la traducción, glosarios, TM.
Definir los roles.

\end{document}