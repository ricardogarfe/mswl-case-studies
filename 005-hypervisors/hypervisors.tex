\documentclass[11pt]{article}
\usepackage[utf8]{inputenc}
\usepackage[spanish]{babel}

\title{\textbf{Hypervisors}}
\author{Ricardo Garc\'ia Fern\'andez}
\date{}
\begin{document}

\maketitle

\section{Introducci\'on}

Hyprevisor es un software que reparte recursos entre las máquinas virtuales.

Existen dos tipos, \textbf{Type-1} y \textbf{Type-2} que veremos a continuación.

\section{Type-1}

Type-1 - corre en el mismo hardware de inicio, es decir está incluido.

Cuando la aplicación hace una llamada al sistema, el hypervisor la captura para el tratamiento y decide que sistema operativo ha de efectuar la llamada al sistema y de esta manera devuelve el resultado al mismo hypervisor.

Guest app - Guest OS - Type-1 - Processor.

\section{Type-2}

Type-2 - aplicación para crear máquinas virtuales dentro del sistema operativo.

El ejemplo práctico es la instalación de un VirtualBox. Por lo que:

    Guest app - Guest OS - Type-2 - Host OS - Processor.

Un capa añadida para la gestión de las llamadas al sistema por lo que el modelo parece tener más carga que el modelo Type-1.

\section{Hypervisor products}

\begin{itemize}
    \item Xen: GPLv2.
    \item KVM: GPLv2, LGPLv2, LGPL, GPL.
    \item VirtualBox: GPLv2. 
\end{itemize}

\subsection{Xen - Citrix}

Comenzó en una universidad (Buscar información), hypervisor Type-1:

\begin{itemize}
    \item Type-1.
    \item Domo0, control, domain. (VM) de control de los nodos de las máquinas a gestionar.
    \item Domo1, unpriviled, domain. (guests)
    \item Hypervisor
    \item Hardware
\end{itemize}

Requiere que el kernel de la máquina comprenda que tiene un hypervisor instalado.

No es necesario emular hardware por lo que se puede sacar partido a un pc antiguo.

\subsection{KVM - RedHat}

Virtualización completa Type-1. A diferencia de Xen, funciona como módulo del kernel por lo que facilita la integración y no ha de entorpecer el desarrollo del sistema operativo para sacar una nueva versión.

\subsection{Qemu}

Aceleración de la emulación de hardware pero no optimiza las instrucciones del kernel.

\subsection{VirtualBox}

Type-2 hypervisor. En tiempo de ejecución se puede recompilar código mediante Qemu para el arranque del sistema (investigar y saber que es esto).

\section{Herramintas de gestión}

\begin{itemize}
    \item Libvirt; redes virtuales, máquinas virtuales, almacenamiento, dispositivos a través de la consola de administración \textbf{Virsh}.
    \item OpenStack; Nova (computo), Swift (almacenamiento, la propia máquina virtual en si isos), Glance (VM repository), Horizon como interfaz de Openstack.. Progrmado en Python. Insfraestructura como servicio.
\end{itemize}

\end{document}
