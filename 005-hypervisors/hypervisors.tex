\documentclass[11pt]{article}
\usepackage[utf8]{inputenc}
\usepackage[parfill]{parskip}
\usepackage{graphicx}
\usepackage{float}
\usepackage{fancyvrb}
\usepackage{listings}

\title{\textbf{Hypervisors}}
\author{Ricardo Garc\'ia Fern\'andez}
\date{}
\begin{document}

\maketitle

\section{Introducci\'on}

\emph{Hyprevisor} es un software que reparte recursos entre las m\'aquinas virtuales.

\par Existen dos tipos, \textbf{Type-1} y \textbf{Type-2} que veremos a continuaci\'on.

\section{Type-1}

Type-1 - corre en el mismo hardware de inicio, es decir est\'a incluido.

Cuando la aplicaci\'on hace una llamada al sistema, el hypervisor la captura para el tratamiento y decide que sistema operativo ha de efectuar la llamada al sistema y de esta manera devuelve el resultado al mismo hypervisor.

Guest app - Guest OS - Type-1 - Processor.

\section{Type-2}

\par Type-2 - aplicaci\'on para crear m\'aquinas virtuales dentro del sistema operativo.

\par El ejemplo pr\'actico es la instalaci\'on de un VirtualBox\footnote{https://www.virtualbox.org/}. Por lo que:

    \emph{Guest app - Guest OS - Type-2 - Host OS - Processor}

\par Un capa a\~nadida para la gesti\'on de las llamadas al sistema por lo que el modelo parece tener m\'as carga que el modelo Type-1.

\section{Hypervisor products}

Una lista detallada de los tipos de \emph{Hypervisor} m\'as conocidos en el mercado:
\begin{itemize}
    \item Xen: GPLv2.
    \item KVM: GPLv2, LGPLv2, LGPL, GPL.
    \item VirtualBox: GPLv2. 
\end{itemize}

\subsection{Xen - Citrix}

Comenz\'o en la universidad de Cambridge y se trata de un hypervisor \emph{Type-1}:

\begin{itemize}
    \item Type-1.
    \item Domo0, control, domain. (VM) de control de los nodos de las m\'aquinas a gestionar.
    \item Domo1, unpriviled, domain. (guests)
    \item Hypervisor
    \item Hardware
\end{itemize}

\par Requiere que el kernel de la m\'aquina comprenda que tiene un hypervisor instalado.

\par No es necesario emular hardware por lo que se puede sacar partido a un pc antiguo.

\subsection{KVM - RedHat}

\par Virtualizaci\'on completa Type-1 del n\'ucleo de Linux. A diferencia de Xen, funciona como m\'odulo del kernel por lo que facilita la integraci\'on y no ha de entorpecer el desarrollo del sistema operativo para sacar una nueva versi\'on.

\par Detr\'as de esta herramienta se encuentra RedHat que previamente utilizaba Xen\footnote{http://www.redhat.com/about/news/archive/2012/9/red-hats-kvm-hypervisor-achieves-top-virtualization-performance-results-with-ibm}.

\subsection{Qemu}

\par \emph{Qemu}\footnote{http://wiki.qemu.org/Main\_Page} nos aporta la aceleraci\'on de la emulaci\'on de hardware pero no optimiza las instrucciones del kernel.

\par Est\'a publicado bajo la licencia GPLv2, aunque tambi\'en aclara que hay m\'odulos que no est\'an licenciado con GPLv2 pero son compatibles\footnote{http://wiki.qemu.org/License}. Est\'a bajo la tutela de Bellard\footnote{http://bellard.org/}

\subsection{VirtualBox}

\par \emph{VirtualBox} es un Type-2 hypervisor.

\par Destaca que en tiempo de ejecuci\'on se puede recompilar c\'odigo mediante Qemu para el arranque del sistema.

\section{Herramintas de gesti\'on}

\par Para el uso en un nivel superior de \'estas soluciones, existen unas herramientas de gesti\'on que facilitan la interacci\'on con los hypervisors:

\begin{itemize}
    \item \emph{Libvirt}; redes virtuales, m\'aquinas virtuales, almacenamiento, dispositivos a trav\'es de la consola de administraci\'on \textbf{Virsh}.
    \item \emph{OpenStack}; Nova (computo), Swift (almacenamiento, la propia m\'aquina virtual en si, isos), Glance (VM repository), Horizon como interfaz de Openstack. Ha sido Programado en Python. Infraestructura como servicio - \emph{IaaS}.
\end{itemize}

\end{document}