\documentclass[11pt]{article}
\usepackage[utf8]{inputenc}
\usepackage[parfill]{parskip}
\usepackage{graphicx}
\usepackage{float}
\usepackage{fancyvrb}
\usepackage{listings}

\title{\textbf{Organize Developer Surveys}}
\author{Ricardo Garc\'ia Fern\'andez}
\date{\today}
\begin{document}

\maketitle

\section{Introducci\'on}

\par Encuestas para definir el perfil de desarrolladores de Software Libre.

\par El desarrollo de software FLOSS entre comunidades de desarrolladores es un enigma a los ojos del p\'ublico. ¿ Como una persona va a aportar su conocimiento (podr\'iamos llamarlo mejor tiempo) a un proyecto a cambio de nada ? No son ideas desacabelladas ni revolucionarias, esta forma de trabajo comunitario no la ha inventado el FLOSS. Aportar por el bien com\'un o s\'implemente, aportar porque solucionando mi problema, puedo solucionar el mismo problema a otra persona. Es algo natural en el ser humano pero hoy en d\'ia, nos parece sorprendente en la sociedad en la que vivimos.

\section{FLOSS study}

\par Los últimos datos recogidos por este tipo de encuestas son hace diez años, en 2002; \emph{FLOSS Study and Survey}\footnote{http://flossproject.org/report/}.

\par 2784 desarrolladores FLOSS fueron encuestados en donde se indagaba sobre los motivos del desarrollador para participar en proyectos FLOSS, la edad, el sexo, los estudios, nacionalidad para intentar dibujar un perfil de las personas que estaban detrás de este tipo de proyectos.

\section{Encuesta}

\par La encuesta desarrollada intenta recoger datos de los desarrolladores de software con respecto a su edad, motivaciones, sexo, conocimientos, estudios, territorio. Para de esta manera poder trazar un perfil a partir de los datos obtenidos y estudiarlos como suma de todos esos datos, intentando de alguna forma ver los puntos en com\'un formando un patr\'on reconocible.

\par Est\'a formada por distintos apartados.
\begin{itemize}

	\item Personal Features of FLOSS Developers
        \begin{itemize}
	        \item Gender and Age
	        \item Partnership / Family Background
	        \item Educational Level of OS/FS Developers
	        \item Professional Background
	        \item Employment Status
	        \item Income
	        \item Nationality, Residence, and Mobility Patterns of OS/FS Developers
        \end{itemize}

	\item Organisational and Work Characteristics
        \begin{itemize}
	        \item Patterns of Time Spending for OS/FS Development
	        \item Preferences of Work Areas and Working Tools
	        \item Project Involvement
	        \item Project Leadership
	        \item Contacts and Central Players within the OS/FS Community
        \end{itemize}

	\item Motivations, Expectations, and Orientations of OS/FS Developers
        \begin{itemize}
	        \item The Social and Political Dimensions of the OS/FS Community
	        \item Motivations for developing Open Source / Free Software
	        \item Expectations Related to the OS/FS Community
        \end{itemize}
    
    \item Dividing Lines
        \begin{itemize}
	        \item Free Software versus Open Source Software
	        \item Open Source / Free Software versus Proprietary Software
	        \item Monetary versus Non-Monetary Rewards
        \end{itemize}

\end{itemize}

\section{Peque\~nas conclusiones}

\par Esta encuesta es un trabajo muy espec\'ifico y muy t\'ecnico para el caso en el que nos encontramos. Las redacci\'on de las conclusiones se puede encontrar en el documento publicado en Google Scholar\footnote{http://scholar.google.es/citations?view\_op=view\_citation\&hl=es\&user=BhVjp-UAAAAJ\&citation\_for\_view=BhVjp-UAAAAJ:u5HHmVD\_uO8C} en el perfil de Gregorio Robles.

\par No se pretende dar una lectura absoluta del documento, s\'implemente pasar por encima de un par de aspectos representados por los resultados de la encuesta.

\par A grandes rasgos se aprecia que la gran mayor\'ia de desarrolladores FLOSS tiene una vida \'util de 10 a\~nos comprendida entre los 18 a 28. Aqu\'i es donde se puede apreciar el contacto con la universidad, las tecnolog\'ias y el acceso a internet. Todo esto conlleva a que est\'an relacionados con el mundo de las ingenier\'ias t\'ecnicas.

\par Por otra parte, vemos que en relaci\'on al g\'enero la participaci\'on masculina es abrumadora. En las relaciones personales vemos que m\'as de la mitad tiene pareja. Esto sirve para derrumbar el t\'ipoco mito de la soledad del desarrollador FLOSS.

\par Como \'ultimo punto a tratar podemos decir que un 22.5\% de los desarrolladores FLOSS dedican menos de un par de horas a la semana, el 26.5\% entre 2 y 5 y el 20.9\% entre 6 y 10. Estos tres grupos se pueden considerar la mayor\'ia pudiendo concretar que la mayor\'ia de desarrolladores FLOSS dedica el tiempo que puede cada semana, poco con respecto al que se dedica al software propietario.

\end{document}
